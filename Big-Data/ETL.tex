% Created 2017-05-30 mar 09:04
% Intended LaTeX compiler: pdflatex
\documentclass[11pt]{article}
\usepackage[utf8]{inputenc}
\usepackage[T1]{fontenc}
\usepackage{graphicx}
\usepackage{grffile}
\usepackage{longtable}
\usepackage{wrapfig}
\usepackage{rotating}
\usepackage[normalem]{ulem}
\usepackage{amsmath}
\usepackage{textcomp}
\usepackage{amssymb}
\usepackage{capt-of}
\usepackage{hyperref}
\author{Jacinto Carrasco Castillo}
\date{2 de junio de 2017}
\title{ETL\\\medskip
\large Big Data y Cloud Computing}
\hypersetup{
 pdfauthor={Jacinto Carrasco Castillo},
 pdftitle={ETL},
 pdfkeywords={},
 pdfsubject={},
 pdfcreator={Emacs 25.2.1 (Org mode 9.0.7)}, 
 pdflang={English}}
\begin{document}

\maketitle
\section{Dataset}
\label{sec:org41b9afe}

El conjunto de datos utilizado ha sido el data set de la competición
\emph{KDD Cup 2010: Student performance evaluation}
\cite{stamper2010algebra}, consistente en los resultados obtenidos por
diferentes alumnos en varios problemas de álgebra. La introducción a
la competición y una descripción del conjunto de datos están
disponibles en la \href{http://www.kdd.org/kdd-cup/view/kdd-cup-2010-student-performance-evaluation/}{web de KDD}, aunque el enlace al propio conjunto está
roto y lo que se ha obtenido ha sido un subconjunto a través del
siguiente \href{https://github.com/Microsoft/azure-docs/blob/master/articles/machine-learning/machine-learning-use-sample-datasets.md}{repositorio} en formato TSV. Al final encontramos el conjunto
de datos \href{https://azuremlsampleexperiments.blob.core.windows.net/datasets/student\_performance.txt}{\texttt{student\_performance.txt}} con el que hemos trabajado aquí.

\section{Consultas}
\label{sec:orgbfe150c}
Comenzamos cargando el conjunto de datos, indicando el tipo de dato
para cada campo. Filtramos la primera entrada pues contiene el nombre
de los campos en el conjunto original.

\begin{verbatim}
algebra = load 'input/student_performance.txt' using PigStorage('\t') 
	AS (Row:int,Anon_Student_Id:chararray,Problem_Hierarchy:chararray,
	Problem_Name:chararray, Problem_View:int, Step_Name:chararray, 
	Step_Start_Time:chararray, First_Transaction_Time:chararray, 
	Correct_Transaction_Time:chararray, Step_End_Time:chararray,
	Step_Duration:int,Correct_Step_Duration:int, Error_Step_Duration:int,
	Correct_First_Attempt:int, Incorrects:int, Hints:int,Corrects:int,
	KC_SubSkills:chararray,Opportunity_SubSkills:chararray,
	KC_KTracedSkills:chararray,Opportunity_KTracedSkills:chararray,
	KC_Rules:chararray,Opportunity_Rules:chararray);
algebra = Filter algebra BY Anon_Student_Id != 'Anon Student Id';
\end{verbatim}

\begin{itemize}
\item \textbf{Operación de proyección}
\end{itemize}

Proyección por el nombre dado al problema para cada entrada. 

\begin{verbatim}
ProblemNames = foreach algebra generate Problem_Name;
\end{verbatim}

\begin{itemize}
\item \textbf{Operación de selección}

  Selección de los problemas resueltos sin fallos y proyección por
la variable identificación del estudiante y por el número de pistas
necesitadas.
\end{itemize}

\begin{verbatim}
CorrectResolutions = FOREACH (FILTER algebra By Incorrects == 0) 
	GENERATE Anon_Student_Id, Hints;
\end{verbatim}

\begin{itemize}
\item \textbf{Agrupamiento y resúmenes de información}

Conteo de problemas resueltos a la primera por cada estudiante y
ordenación posterior según el número de pistas totales que
necesitaron.
\end{itemize}

\begin{verbatim}
GroupsByStudent = GROUP CorrectResolutions BY  Anon_Student_Id;
CountedStudent = FOREACH GroupsByStudent GENERATE 
	group AS Anon_Student_Id, 
	COUNT(CorrectResolutions) AS count, 
	SUM(CorrectResolutions.Hints) AS sum_hints;
OrderedStudents = ORDER CountedStudent BY count DESC, sum_hints ASC;
\end{verbatim}

\section{Referencias}
\label{sec:orgf29b0b6}
\bibliographystyle{plain}
\bibliography{bibliography}
\end{document}