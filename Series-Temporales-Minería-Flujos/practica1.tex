% Created 2017-05-03 mié 15:30
% Intended LaTeX compiler: pdflatex
\documentclass[11pt]{article}
\usepackage[utf8]{inputenc}
\usepackage[T1]{fontenc}
\usepackage{graphicx}
\usepackage{grffile}
\usepackage{longtable}
\usepackage{wrapfig}
\usepackage{rotating}
\usepackage[normalem]{ulem}
\usepackage{amsmath}
\usepackage{textcomp}
\usepackage{amssymb}
\usepackage{capt-of}
\usepackage{hyperref}
\author{Jacinto Carrasco Castillo - jacintocc@correo.ugr.es}
\date{5 de mayo de 2017}
\title{Trabajo autónomo II: Minería de flujo de datos\\\medskip
\large Máster en Ciencia de Datos e Ingeniería de Computadores - Minería de flujo de datos y series temporales}
\hypersetup{
 pdfauthor={Jacinto Carrasco Castillo - jacintocc@correo.ugr.es},
 pdftitle={Trabajo autónomo II: Minería de flujo de datos},
 pdfkeywords={},
 pdfsubject={},
 pdfcreator={Emacs 25.1.1 (Org mode 9.0.5)}, 
 pdflang={English}}
\begin{document}

\maketitle
\tableofcontents



\section{Parte teórica}
\label{sec:org78b2585}

\subsection{Pregunta 1}
\label{sec:org9fb51c3}

Descripción del problema de clasificación, los clasificadores
utilizados en los experimentos de la sección 2, y en qué consisten los
diferentes modos de evaluación/validación en flujos de datos.

\subsection{Pregunta 2}
\label{sec:orgd52cc5f}

Explicar en qué consiste el problema de \textbf{\textbf{concept drift}}
y qué técnicas conoce para resolverlo en clasificación.




\section{Parte práctica}
\label{sec:org048e30d}

\subsection{Entrenamiento offline y evaluación posterior}
\label{sec:orgf531c4d}
\subsubsection{Hoeffding Tree}
\label{sec:orgf0410a7}
\begin{itemize}
\item Entrenar un clasificador \texttt{HoeffdingTree} \emph{offline} (aprender modelo
únicamente),sobre un total de 1.000.000 de instancias procedentes de
un flujo obtenido por el generador \texttt{WaveFormGenerator} con semilla
aleatoria igual a 2. Evaluar posteriormente (sólo evaluación) con
1.000.000 de instancias generadas por el mismo tipo de generador,
con semilla aleatoria igual a 4. Repita el proceso varias veces con
la misma semilla en evaluación y diferentes semillas en
entrenamiento. Anotar los valores de porcentajes de aciertos en la
clasificación y estadístico Kappa.
\end{itemize}


Realizamos el aprendizaje del clasificador \texttt{HoeffdingTree} con un flujo
generado con semilla \(2\).

\begin{verbatim}
java -cp moa.jar -javaagent:sizeofag.jar moa.DoTask \
  "LearnModel -l trees.HoeffdingTree \
     -s (generators.WaveformGenerator -i 2) \
     -m 1000000 -O model1.moa"
\end{verbatim}
\captionof{figure}{\label{orgbeedfa4}
"Aprendizaje HoeffdingTree offline"}



Una vez mostrado el ejemplo de lo que sería el código, realizaremos la
evaluación de los modelos generados con las semillas
\(\{2,10,15,20,25\}\) sobre un flujo de datos creado con la semilla \(4\).

\begin{verbatim}
for seed in 2 10 15 20 25
do
   java -cp moa.jar -javaagent:sizeofag.jar moa.DoTask \
      "EvaluateModel \
	 -m (LearnModel -l trees.HoeffdingTree \
	    -s (generators.WaveformGenerator -i $seed) -m 1000000) \
	 -s (generators.WaveformGenerator -i 4) -i 1000000"
done
\end{verbatim}
\captionof{figure}{\label{orgbfbdfc9}
"Aprendizaje y evaluación HoeffdingTree"}

Mostramos a continuación el porcentaje de acierto en clasificación
obtenido para cada una de las semillas y la media de éstos.
\subsubsection{Hoeffding Tree Adaptativo}
\label{sec:orgbae41d9}
\begin{itemize}
\item Repetir el paso anterior, sustituyendo el clasificador por
HoeffdingTree adaptativo.
\end{itemize}

Realizamos directamente el aprendizaje para las semillas anteriores y
la evaluación sobre el flujo generado con semilla 2.

\begin{verbatim}
for seed in 2 10 15 20 25
do
   java -cp moa.jar -javaagent:sizeofag.jar moa.DoTask \
      "EvaluateModel \
	 -m (LearnModel -l ntrees.HoeffdingAdaptiveTree
	    -s (generators.WaveformGenerator -i $seed) -m 1000000) \
	 -s (generators.WaveformGenerator -i 4) -i 1000000"
done
\end{verbatim}
\captionof{figure}{\label{org15830e8}
"Aprendizaje y evaluación HoeffdingTree Adaptativo"}

Mostramos la tabla con las medidas obtenidas por este clasificador. 

\begin{itemize}
\item Responda a la pregunta: ¿Cree que algún clasificador es
significativamente mejor que el otro en este tipo de problemas?
Razone su respuesta.
\end{itemize}

Nope. 

\subsection{Entrenamiento online}
\label{sec:org7505d75}

\subsubsection{Hoeffding Tree}
\label{sec:org69bb4cb}
\begin{itemize}
\item Entrenar un clasificador HoeffdingTree online, mediante el método
Interleaved Test-Then-Train, sobre un total de 1.000.000 de
instancias procedentes de un flujo obtenido por el generador
WaveFormGenerator con semilla aleatoria igual a 2, con una
frecuencia de muestreo igual a 10.000. Pruebe con otras semillas
aleatorias. Anotar los valores de porcentajes de aciertos en la
clasificación y estadístico Kappa.
\end{itemize}

Para usar el método \texttt{EvaluateInterleavedTestThenTrain} incluimos el
número de instancias pasándole el argumento \texttt{-i} y la frecuencia de
muestreo con \texttt{-f}.

\begin{verbatim}
for seed in 2 10 15 20 25 
do 
 java -cp moa.jar -javaagent:sizeofag.jar moa.DoTask "EvaluateInterleavedTestThenTrain \
    -l moa.classifiers.trees.HoeffdingTree \
    -s (generators.WaveformGenerator -i $seed) -i 1000000 -f 10000"
done
\end{verbatim}

Mostramos la tabla con los resultados obtenidos por el clasificador
Hoeffding Tree en línea. 

% latex table generated in R 3.3.2 by xtable 1.8-2 package
% Wed May  3 09:58:35 2017
\begin{table}[ht]
\centering
\begin{tabular}{rlrr}
  \hline
 & Seed & Acc & Kappa \\ 
  \hline
1 & 2 & 83.79 & 75.68 \\ 
  2 & 10 & 83.85 & 75.77 \\ 
  3 & 15 & 83.97 & 75.95 \\ 
  4 & 20 & 83.79 & 75.69 \\ 
  5 & 25 & 83.93 & 75.89 \\ 
  6 & media & 83.86 & 75.80 \\ 
   \hline
\end{tabular}
\end{table}

\subsubsection{Hoeffding Tree Adaptativo}
\label{sec:orgce49934}
\begin{itemize}
\item Repetir el paso anterior, sustituyendo el clasificador por
HoeffdingTree adaptativo.
\end{itemize}


\begin{verbatim}
for seed in 2 10 15 20 25 
do 
 java -cp moa.jar -javaagent:sizeofag.jar moa.DoTask \
    "EvaluateInterleavedTestThenTrain \
      -l moa.classifiers.trees.HoeffdingAdaptiveTree \
	-s (generators.WaveformGenerator -i $seed) \
	-i 1000000 -f 10000"
done
\end{verbatim}
\captionof{figure}{\label{orgd2370f0}
"Aprendizaje Online Hoeffding Tree Adaptativo"}

% latex table generated in R 3.3.2 by xtable 1.8-2 package
% Wed May  3 12:48:05 2017
\begin{table}[ht]
\centering
\begin{tabular}{rlrr}
  \hline
 & Seed & Acc & Kappa \\ 
  \hline
1 & 2 & 83.73 & 75.60 \\ 
  2 & 10 & 83.90 & 75.85 \\ 
  3 & 15 & 83.87 & 75.80 \\ 
  4 & 20 & 83.81 & 75.72 \\ 
  5 & 25 & 83.83 & 75.74 \\ 
  6 & media & 83.83 & 75.74 \\ 
   \hline
\end{tabular}
\end{table}

\subsubsection{Comparación}
\label{sec:orga197d12}
\begin{itemize}
\item Responda a la pregunta: ¿Cree que algún clasificador es mejor que el
otro en este tipo de problemas? Razone su respuesta.
\end{itemize}

Nope.

\subsection{Entrenamiento online en datos con \emph{concept drift}.}
\label{sec:org854545b}

\subsubsection{Hoeffding Tree}
\label{sec:orgeccf6ec}
\begin{itemize}
\item Entrenar un clasificador HoeffdingTree online, mediante el método
Interleaved Test-Then-Train, sobre un total de 2.000.000 de
instancias muestreadas con una frecuencia de 100.000, sobre datos
procedentes de un generador de flujos RandomRBFGeneratorDrift, con
semilla aleatorio igual a 1 para generación de modelos y de
instancias, generando 2 clases, 7 atributos, 3 centroides en el
modelo, drift en todos los centroides y velocidad de cambio igual a
0.001. Pruebe con otras semillas aleatorias. Anotar los valores de
porcentajes de aciertos en la clasificación y estadístico
Kappa. Compruebe la evolución de la curva de aciertos en la GUI de
MOA.
\end{itemize}

\begin{verbatim}
for seed in 1 8 10 53 61
do 
 java -cp moa.jar -javaagent:sizeofag.jar moa.DoTask \
    "EvaluateInterleavedTestThenTrain \
      -l moa.classifiers.trees.HoeffdingTree \
	-s (generators.RandomRBFGeneratorDrift \
	       -r $seed -i $seed -s 0.001 -k 3 -a 7 -n 3) \
	-i 2000000 -f 100000"
done
\end{verbatim}
\captionof{figure}{\label{org450ca75}
"Aprendizaje Online Hoeffding Tree - Concept Drift"}

\subsubsection{Adaptativo}
\label{sec:org1b9c2e4}

\begin{verbatim}
for seed in 1 8 10 53 61
do 
 java -cp moa.jar -javaagent:sizeofag.jar moa.DoTask \
    "EvaluateInterleavedTestThenTrain \
      -l moa.classifiers.trees.HoeffdingAdaptiveTree \
	-s (generators.RandomRBFGeneratorDrift \
	       -r $seed -i $seed -s 0.001 -k 3 -a 7 -n 3) \
	-i 2000000 -f 100000"
done
\end{verbatim}
\captionof{figure}{\label{org22efcf8}
"Aprendizaje Online Hoeffding Tree Adaptativo - Concept Drift"}

\subsubsection{Comparación}
\label{sec:orgf5606ea}

Aquí claro que sí


\subsection{Entrenamiento online en datos con \emph{concept drift}, incluyendo mecanismos para olvidar instancias pasadas.}
\label{sec:org8896302}

\subsubsection{Ventana deslizante}
\label{sec:org16f8ba5}
\begin{itemize}
\item Repita la experimentación del apartado anterior, cambiando el método
de evaluación “Interleaved Test-Then-Train” por el método de
evaluación “Prequential”, con una ventana deslizante de tamaño 1.000.
\end{itemize}

\begin{enumerate}
\item Hoeffding
\label{sec:orgecbe67c}

\item Hoeffding Adaptativo
\label{sec:orgb638d5d}
\end{enumerate}

\subsubsection{Comparación}
\label{sec:org78c3044}
\begin{itemize}
\item ¿Qué efecto se nota en ambos clasificadores? ¿A qué es debido?
Justifique los cambios relevantes en los resultados de los
clasificadores
\end{itemize}

\subsection{Entrenamiento online en datos con concept drift, incluyendo mecanismos para reinicializar modelos tras la detección de cambios de concepto.}
\label{sec:org98e764e}

\begin{itemize}
\item Repita la experimentación del apartado 2.3, cambiando el modelo
(learner) a un clasificador simple basado en reemplazar el
clasificador actual cuando se detecta un cambio de concepto
(SingleClassifierDrift). Como detector de cambio de concepto, usar
el método DDM con sus parámetros por defecto. Como modelo a
aprender, usar un clasificador HoeffdingTree.

\item Repita el paso anterior cambiando el clasificador HoeffdingTree por
un clasificador HoeffdingTree adaptativo.

\item Responda a la siguiente pregunta: ¿Qué diferencias se producen entre
los métodos de los apartados 2.3, 2.4 y 2.5? Explique similitudes y
diferencias entre las diferentes metodologías, y discuta los
resultados obtenidos por cada una de ellas en el flujo de datos
propuesto.
\end{itemize}
\end{document}
