% Created 2017-05-05 vie 15:24
% Intended LaTeX compiler: pdflatex
\documentclass[11pt]{article}
\usepackage[utf8]{inputenc}
\usepackage[T1]{fontenc}
\usepackage{graphicx}
\usepackage{grffile}
\usepackage{longtable}
\usepackage{wrapfig}
\usepackage{rotating}
\usepackage[normalem]{ulem}
\usepackage{amsmath}
\usepackage{textcomp}
\usepackage{amssymb}
\usepackage{capt-of}
\usepackage{hyperref}
\author{Jacinto Carrasco Castillo - jacintocc@correo.ugr.es}
\date{5 de mayo de 2017}
\title{Trabajo autónomo II: Minería de flujo de datos\\\medskip
\large Máster en Ciencia de Datos e Ingeniería de Computadores - Minería de flujo de datos y series temporales}
\hypersetup{
 pdfauthor={Jacinto Carrasco Castillo - jacintocc@correo.ugr.es},
 pdftitle={Trabajo autónomo II: Minería de flujo de datos},
 pdfkeywords={},
 pdfsubject={},
 pdfcreator={Emacs 25.1.1 (Org mode 9.0.5)}, 
 pdflang={English}}
\begin{document}

\maketitle
\setcounter{tocdepth}{2}
\tableofcontents


\section{Parte teórica}
\label{sec:orgfdc5129}

\subsection{Pregunta 1}
\label{sec:org86efbe3}

Explicar el problema de clasificación, los clasificadores utilizados
 en los experimentos de la sección 2, y en qué consisten los
 diferentes modos de evaluación/validación en  flujos de datos.

Los clasificadores utilizados han sido el \texttt{Hoeffding Tree} y el
\texttt{Hoeffding Tree Adaptativo}. El clasificador \texttt{Hoeffding Tree} se basa
en la idea de que una pequeña muestra puede ser suficiente para
escoger un atributo que nos permita escoger la clase de la instancia,
para lo que se asume que la distribución de datos generados durante el
flujo no varía en el tiempo. La versión adaptativa de este
clasificador usa \texttt{ADWIN} para comprobar el rendimiento de las rámas
del árbol para irlas modificando con nuevas ramas cuando su porcentaje
de acierto disminuye si las nuevas ramas son más precisas.

Los modos de evaluación en flujo de datos son:
\begin{itemize}
\item Test-Then-Train, donde se usa cada nuevo dato para evaluar
el modelo e inmediatamente después es usado para
entenamiento;
\item Prequential evaluation, que incluye un mecanismo de ventana
deslizante para olvidar los datos antiguos.
\item Heldout: Se reserva un número de datos para test y se
entrena con el resto.
\end{itemize}

\subsection{Pregunta 2}
\label{sec:org34b12da}

Explicar en qué consiste el problema de \emph{concept drift} y qué
 técnicas conoce para resolverlo en clasificación.


Nos referimos con \emph{Concept drift} a las situaciones en las que el
fenómeno estudiado sufre variaciones en sus propiedades estadísticas a
lo largo del tiempo, por lo que debemos ser capaces de detectar estas
variaciones y reaprender un modelo para ajustarnos a estas nuevas
circunstancias. 

\section{Parte práctica}
\label{sec:org6b7d2d7}

\subsection{Entrenamiento offline y evaluación posterior}
\label{sec:org4c77238}
\subsubsection{Hoeffding Tree}
\label{sec:org7abd73d}
Entrenar un clasificador \texttt{HoeffdingTree} \emph{offline} (aprender modelo
únicamente),sobre un total de 1.000.000 de instancias procedentes de
un flujo obtenido por el generador \texttt{WaveFormGenerator} con semilla
aleatoria igual a 2. Evaluar posteriormente (sólo evaluación) con
1.000.000 de instancias generadas por el mismo tipo de generador,
con semilla aleatoria igual a 4. Repita el proceso varias veces con
la misma semilla en evaluación y diferentes semillas en
entrenamiento. Anotar los valores de porcentajes de aciertos en la
clasificación y estadístico Kappa.


Realizamos el aprendizaje del clasificador \texttt{HoeffdingTree} con un flujo
generado con semilla \(2\).

\begin{verbatim}
java -cp moa.jar -javaagent:sizeofag.jar moa.DoTask \
  "LearnModel -l trees.HoeffdingTree \
     -s (generators.WaveformGenerator -i 2) \
     -m 1000000 -O model1.moa"
\end{verbatim}
\captionof{figure}{\label{orge32ec9a}
"Aprendizaje HoeffdingTree"}


Una vez mostrado el ejemplo de lo que sería el código, realizaremos la
evaluación de los modelos generados con las semillas
\(\{2,10,15,20,25\}\) sobre un flujo de datos creado con la semilla \(4\).

\begin{verbatim}
for seed in 2 10 15 20 25
do
   java -cp moa.jar -javaagent:sizeofag.jar moa.DoTask \
      "EvaluateModel \
	 -m (LearnModel -l trees.HoeffdingTree \
	    -s (generators.WaveformGenerator -i $seed) -m 1000000) \
	 -s (generators.WaveformGenerator -i 4) -i 1000000"
done
\end{verbatim}
\captionof{figure}{\label{org3cdcdb0}
"Aprendizaje y evaluación HoeffdingTree con varias semillas"}

Mostramos a continuación el porcentaje de acierto en clasificación
obtenido para cada una de las semillas y la media de éstos.
% latex table generated in R 3.3.2 by xtable 1.8-2 package
% Thu May  4 16:07:07 2017
\begin{table}[ht]
\centering
\begin{tabular}{rlrr}
  \hline
 & Seed & Acc & Kappa \\ 
  \hline
1 & 2 & 84.51 & 76.77 \\ 
  2 & 10 & 84.58 & 76.87 \\ 
  3 & 15 & 84.65 & 76.97 \\ 
  4 & 20 & 84.57 & 76.85 \\ 
  5 & 25 & 84.65 & 76.97 \\ 
  6 & media & 84.59 & 76.89 \\ 
   \hline
\end{tabular}
\end{table}

\subsubsection{Hoeffding Tree Adaptativo}
\label{sec:org49f8971}
\begin{itemize}
\item Repetir el paso anterior, sustituyendo el clasificador por
HoeffdingTree adaptativo.
\end{itemize}

Realizamos directamente el aprendizaje para las semillas anteriores y
la evaluación sobre el flujo generado con semilla 2.

\begin{verbatim}
for seed in 2 10 15 20 25
do
   java -cp moa.jar -javaagent:sizeofag.jar moa.DoTask \
      "EvaluateModel \
	 -m (LearnModel -l ntrees.HoeffdingAdaptiveTree
	    -s (generators.WaveformGenerator -i $seed) -m 1000000) \
	 -s (generators.WaveformGenerator -i 4) -i 1000000"
done
\end{verbatim}
\captionof{figure}{\label{org8efeee8}
"Aprendizaje y evaluación HoeffdingTree Adaptativo"}

\begin{verbatim}
library(xtable)
seeds <- c(2,10,15,20,25)
acc <- x[seq(2, by = 14, length.out = 5), 5]
acc <- as.numeric(gsub(",",".",acc))
kappa <- x[seq(3, by = 14, length.out = 5), 5]
kappa <- as.numeric(gsub(",",".",kappa))

df <- data.frame("Seed" = c(seeds,"media"), 
		 "Acc" = c(acc,mean(acc)),
		 "Kappa" = c(kappa,mean(kappa)))
xtable(df)
\end{verbatim}

Mostramos la tabla con las medidas obtenidas por este clasificador. 

\subsubsection{Comparación}
\label{sec:orgfaa1a81}
\begin{itemize}
\item Responda a la pregunta: ¿Cree que algún clasificador es
significativamente mejor que el otro en este tipo de problemas?
Razone su respuesta.
\end{itemize}

Para realizar la comparación de los dos métodos realizaremos un test
estadístico sobre las cinco muestras obtenidas. El resultado que
esperamos obtener es que no haya una diferencia significativa entre
los algoritmos, ya que el aprendizaje se realiza \emph{off line} a partir
del conjunto total de datos. 

Como vemos, los resultados son muy similares, y según un test de
Wilcoxon aplicado sobre estos valores no podemos descartar que el
rendimiento sea equivalente.


	\underline{Wilcoxon rank sum test}\\

data:  $x$ and $y$ \\
W = 22, p-value = 0.05556 \\
alternative hypothesis: true location shift is not equal to $0$ \\


\subsection{Entrenamiento online}
\label{sec:org79a301f}

Los experimentos de los siguientes apartados se harán en línea de
comandos y volcaremos los resultados en ficheros \texttt{.csv} para realizar
posteriormente las gráficas y test estadísticos para las comparaciones
en \texttt{R}. Para ello definimos una función que nos recupere la
información relevante de los archivos \texttt{.csv}. Las variables en las que
nos fijaremos serán en el número de instancias, el porcentaje de
acierto, el estadístico Kappa, el número de nodos y la profundidad del
árbol. Para la comparación de cada apartado entre el modelo de
\texttt{HoeffdingTree} y el modelo que incluye adaptación, nos fijaremos en
la última iteración y en la media por iteraciones, para promediar
finalmente por la semilla. 

\begin{verbatim}
readMoaOutput <- function(dir){
   files <- list.files(dir, full.names = T)
   lapply(files, function(f){
      info <- read.csv(f)[ ,c(1,5,6,11,14)]
      colnames(info) <- c("Instances", "Acc", "Kappa", "Nodes", "Depth")
      return(info)
   })
}

summaryMoaSeeds <- function(list.results){
   summary <- t(sapply(list.results,
	  function(seed.results){
	  return(t(matrix(c(seed.results[nrow(seed.results),-1], 
			apply(seed.results[ ,-1], 2, mean)), ncol = 2)))
	  }))
   summary <- matrix(unlist(summary), ncol = 8)
   colnames(summary) <- paste(c("Last","Mean"),
			      rep(c("Acc","Kappa","Nodes","Depth"),each=2),
			      sep=".")
   return(summary)
}

summaryMoa <- function(list.results){
   summary <- summaryMoaSeeds(list.results)
   summary <- matrix(apply(summary, 2, mean), ncol = 4)   
   colnames(summary) <- c("Acc","Kappa","Nodes","Depth")
   rownames(summary) <- c("Last", "Mean")
   return(summary)
}
\end{verbatim}


\subsubsection{Hoeffding Tree}
\label{sec:orgcd9aaf0}
\begin{itemize}
\item Entrenar un clasificador HoeffdingTree online, mediante el método
Interleaved Test-Then-Train, sobre un total de 1.000.000 de
instancias procedentes de un flujo obtenido por el generador
WaveFormGenerator con semilla aleatoria igual a 2, con una
frecuencia de muestreo igual a 10.000. Pruebe con otras semillas
aleatorias. Anotar los valores de porcentajes de aciertos en la
clasificación y estadístico Kappa.
\end{itemize}

Para usar el método \texttt{EvaluateInterleavedTestThenTrain} incluimos el
número de instancias pasándole el argumento \texttt{-i} y la frecuencia de
muestreo con \texttt{-f}.

\begin{verbatim}
for seed in 2 10 15 20 25 
do 
 java -cp moa.jar -javaagent:sizeofag.jar moa.DoTask \
   "EvaluateInterleavedTestThenTrain \
    -l moa.classifiers.trees.HoeffdingTree \
    -s (generators.WaveformGenerator -i $seed) \
   -i 1000000 -f 10000" > Resultados/Online/Hoeff/hoeff-$seed.csv
done
\end{verbatim}

% latex table generated in R 3.3.2 by xtable 1.8-2 package
% Fri May  5 15:24:20 2017
\begin{table}[ht]
\centering
\begin{tabular}{rrrrr}
  \hline
 & Acc & Kappa & Nodes & Depth \\ 
  \hline
Last & 83.88 & 75.82 & 317.00 & 12.20 \\ 
  Mean & 82.98 & 74.47 & 157.02 & 8.94 \\ 
   \hline
\end{tabular}
\end{table}

\subsubsection{Hoeffding Tree Adaptativo}
\label{sec:orge1a5bea}
\begin{itemize}
\item Repetir el paso anterior, sustituyendo el clasificador por
HoeffdingTree adaptativo.
\end{itemize}


\begin{verbatim}
for seed in 2 10 15 20 25 
do 
 java -cp moa.jar -javaagent:sizeofag.jar moa.DoTask \
    "EvaluateInterleavedTestThenTrain \
      -l moa.classifiers.trees.HoeffdingAdaptiveTree \
	-s (generators.WaveformGenerator -i $seed) \
	-i 1000000 -f 10000" > Resultados/Online/Adap/adap-$seed.csv
done
\end{verbatim}
\captionof{figure}{\label{org8d649f5}
"Aprendizaje Online Hoeffding Tree Adaptativo"}

% latex table generated in R 3.3.2 by xtable 1.8-2 package
% Fri May  5 15:24:21 2017
\begin{table}[ht]
\centering
\begin{tabular}{rrrrr}
  \hline
 & Acc & Kappa & Nodes & Depth \\ 
  \hline
Last & 83.84 & 75.76 & 432.60 & 13.80 \\ 
  Mean & 83.06 & 74.59 & 214.65 & 10.18 \\ 
   \hline
\end{tabular}
\end{table}

\subsubsection{Comparación}
\label{sec:org21ad7a4}
\begin{itemize}
\item Responda a la pregunta: ¿Cree que algún clasificador es mejor que el
otro en este tipo de problemas? Razone su respuesta.

Podemos observar en los resultados medios que no hay diferencias
significativas entre los dos métodos. Si aplicamos el test de
Wilcoxon obtenemos un \$p\$-valor muy superior al nivel de
significación, con lo que no podemos descartar que su acierto sea
idéntico.
\end{itemize}


	underline{Wilcoxon rank sum test with continuity correction}\\

data:  Accuracy in Online Hoeffding Tree and Accuracy in Online Adaptive\\
W = 743, p-value = 0.5866\\
alternative hypothesis: true location shift is not equal to 0\\

\subsection{Entrenamiento online en datos con \emph{concept drift}.}
\label{sec:org377b52c}

\subsubsection{Hoeffding Tree}
\label{sec:orgab33ef8}
\begin{itemize}
\item Entrenar un clasificador HoeffdingTree online, mediante el método
Interleaved Test-Then-Train, sobre un total de 2.000.000 de
instancias muestreadas con una frecuencia de 100.000, sobre datos
procedentes de un generador de flujos RandomRBFGeneratorDrift, con
semilla aleatorio igual a 1 para generación de modelos y de
instancias, generando 2 clases, 7 atributos, 3 centroides en el
modelo, drift en todos los centroides y velocidad de cambio igual a
0.001. Pruebe con otras semillas aleatorias. Anotar los valores de
porcentajes de aciertos en la clasificación y estadístico
Kappa. Compruebe la evolución de la curva de aciertos en la GUI de
MOA.
\end{itemize}

En estos experimentos la semilla también afectará a la generación del
\emph{concept drift}. 

\begin{verbatim}
for seed in 1 2 314 261 832
do
   java -cp moa.jar -javaagent:sizeofag.jar moa.DoTask \
    "EvaluateInterleavedTestThenTrain \
      -l moa.classifiers.trees.HoeffdingTree \
	-s (generators.RandomRBFGeneratorDrift \
	       -r $seed -i $seed -s 0.001 -k 3 -a 7 -n 3) \
	-i 2000000 -f 100000" > Resultados/Drift/Hoeff/hoeff-$seed.csv
done
\end{verbatim}
\captionof{figure}{\label{org177aa74}
"Aprendizaje Online Hoeffding Tree - Concept Drift"}


% latex table generated in R 3.3.2 by xtable 1.8-2 package
% Fri May  5 15:24:22 2017
\begin{table}[ht]
\centering
\begin{tabular}{rrrrr}
  \hline
 & Acc & Kappa & Nodes & Depth \\ 
  \hline
Last & 77.47 & 36.73 & 1875.40 & 16.40 \\ 
  Mean & 79.08 & 41.61 & 1023.70 & 14.36 \\ 
   \hline
\end{tabular}
\end{table}

\subsubsection{Adaptativo}
\label{sec:org1ef858e}

\begin{verbatim}
for seed in 1 2 314 261 832
do 
 java -cp moa.jar -javaagent:sizeofag.jar moa.DoTask \
    "EvaluateInterleavedTestThenTrain \
      -l moa.classifiers.trees.HoeffdingAdaptiveTree \
	-s (generators.RandomRBFGeneratorDrift \
	       -r $seed -i $seed -s 0.001 -k 3 -a 7 -n 3) \
	-i 2000000 -f 100000" > Resultados/Drift/Adap/adap-$seed.csv
done
\end{verbatim}
\captionof{figure}{\label{org91aa6fb}
"Aprendizaje Online Hoeffding Tree Adaptativo - Concept Drift"}


% latex table generated in R 3.3.2 by xtable 1.8-2 package
% Fri May  5 15:24:22 2017
\begin{table}[ht]
\centering
\begin{tabular}{rrrrr}
  \hline
 & Acc & Kappa & Nodes & Depth \\ 
  \hline
Last & 92.93 & 81.17 & 2936.60 & 2.80 \\ 
  Mean & 92.97 & 81.30 & 1558.87 & 2.37 \\ 
   \hline
\end{tabular}
\end{table}


\subsubsection{Comparación}
\label{sec:orga7688d4}

Podemos observar en los resultados medios que no hay diferencias
significativas entre los dos métodos. Si aplicamos el test de
Wilcoxon obtenemos un \$p\$-valor inferior a 0.01, con lo que podemos
rechazar la hipótesis de que sean equivalentes. Si observamos las
tablas de cada modelo, observamos que el \texttt{HoeffdingTree} tiene una
profundidad mucho mayor, lo que significa que el modelo va
aprendiendo pero no va olvidando lo ocurrido anteriormente que ya no
funciona y por tanto sufre con los \emph{concept drifts}.


	\underline{Wilcoxon rank sum test with continuity correction}\\

data:  acc.drift.hoeff and acc.drift.adap\\
W = 0, p-value = 0.01219\\
alternative hypothesis: true location shift is not equal to 0



\subsection{Entrenamiento online en datos con \emph{concept drift}, incluyendo mecanismos para olvidar instancias pasadas.}
\label{sec:org8db1e6b}

\subsubsection{Ventana deslizante}
\label{sec:org1f5439e}
\begin{itemize}
\item Repita la experimentación del apartado anterior, cambiando el método
de evaluación “Interleaved Test-Then-Train” por el método de
evaluación “Prequential”, con una ventana deslizante de tamaño 1.000.
\end{itemize}

\begin{enumerate}
\item Hoeffding
\label{sec:orgd34ad4e}

\begin{verbatim}
for seed in 1 2 314 261 832
do
  java -cp moa.jar -javaagent:sizeofag.jar moa.DoTask \
    "EvaluatePrequential \
       -l trees.HoeffdingTree \
	  -s (generators.RandomRBFGeneratorDrift -s 0.001 \
	       -r $seed -i $seed -s 0.001 -k 3 -a 7 -n 3) \
       -i 2000000" > Resultados/Window/Hoeff/hoeff-$seed.csv
done
\end{verbatim}
\captionof{figure}{\label{org38a3dc1}
"Aprendizaje Online Hoeffding Tree - Concept Drift - Ventana"}

% latex table generated in R 3.3.2 by xtable 1.8-2 package
% Fri May  5 15:24:23 2017
\begin{table}[ht]
\centering
\begin{tabular}{rrrrr}
  \hline
 & Acc & Kappa & Nodes & Depth \\ 
  \hline
Last & 75.04 & 31.35 & 1875.40 & 16.40 \\ 
  Mean & 77.79 & 36.27 & 1023.70 & 14.36 \\ 
   \hline
\end{tabular}
\end{table}

\item Hoeffding Adaptativo
\label{sec:org5f24b72}

\begin{verbatim}
for seed in 1 2 314 261 832
do
  java -cp moa.jar -javaagent:sizeofag.jar moa.DoTask \
    "EvaluatePrequential \
       -l trees.HoeffdingAdaptiveTree \
	  -s (generators.RandomRBFGeneratorDrift -s 0.001 \
	       -r $seed -i $seed -s 0.001 -k 3 -a 7 -n 3) \
       -i 2000000" > Resultados/Window/Adap/adap-$seed.csv
done
\end{verbatim}
\captionof{figure}{\label{orgcef2b38}
"Aprendizaje Online Hoeffding Tree Adaptativo - Concept Drift - Ventana"}

% latex table generated in R 3.3.2 by xtable 1.8-2 package
% Fri May  5 15:24:24 2017
\begin{table}[ht]
\centering
\begin{tabular}{rrrrr}
  \hline
 & Acc & Kappa & Nodes & Depth \\ 
  \hline
Last & 92.96 & 82.37 & 2936.60 & 2.80 \\ 
  Mean & 92.88 & 81.00 & 1558.87 & 2.37 \\ 
   \hline
\end{tabular}
\end{table}
\end{enumerate}

\subsubsection{Comparación}
\label{sec:orgcc235ad}
\begin{itemize}
\item ¿Qué efecto se nota en ambos clasificadores? ¿A qué es debido?
Justifique los cambios relevantes en los resultados de los
clasificadores
\end{itemize}


	Wilcoxon rank sum test with continuity correction

data:  acc.window.hoeff and acc.window.adap
W = 0, p-value = 0.01219
alternative hypothesis: true location shift is not equal to 0


A pesar de incluir el mecanismo para olvidar instancias antiguas, el
\texttt{HoeffdingTree} adaptativo sigue siendo mejor significativamente que
el modelo no adaptativo. Esto se porduce debido a que al haber un
cambio en el flujo de datos, el modelo adaptativo reacciona mejor,
desechando la información aprendida antes de que las instancias de la
situación anterior desaparezcan por antiguas.
\subsection{Entrenamiento online en datos con concept drift, incluyendo mecanismos para reinicializar modelos tras la detección de cambios de concepto.}
\label{sec:org05be011}

\begin{itemize}
\item Repita la experimentación del apartado 2.3, cambiando el modelo
(learner) a un clasificador simple basado en reemplazar el
clasificador actual cuando se detecta un cambio de concepto
(SingleClassifierDrift). Como detector de cambio de concepto, usar
el método DDM con sus parámetros por defecto. Como modelo a
aprender, usar un clasificador HoeffdingTree.
\end{itemize}

\subsubsection{Hoeffding Tree}
\label{sec:org9b293a1}
\begin{verbatim}
for seed in 1 2 314 261 832
do
  java -cp moa.jar -javaagent:sizeofag.jar moa.DoTask \
    "EvaluatePrequential \
      -l (drift.SingleClassifierDrift -l trees.HoeffdingTree) \
	  -s (generators.RandomRBFGeneratorDrift -s 0.001 \
	       -r $seed -i $seed -s 0.001 -k 3 -a 7 -n 3) \
       -i 2000000" > Resultados/Reboot/Hoeff/hoeff-$seed.csv
done
\end{verbatim}
\captionof{figure}{\label{orgab3cb6b}
"Aprendizaje Online - Concept Drift - Window - Reboot"}


% latex table generated in R 3.3.2 by xtable 1.8-2 package
% Fri May  5 15:24:25 2017
\begin{table}[ht]
\centering
\begin{tabular}{rrrrr}
  \hline
 & Acc & Kappa & Nodes & Depth \\ 
  \hline
Last & 90.36 & 68.49 & 41.60 & 69.80 \\ 
  Mean & 92.20 & 75.82 & 52.68 & 24.58 \\ 
   \hline
\end{tabular}
\end{table}


\subsubsection{Hoeffding Tree Adaptativo}
\label{sec:org5d37bcd}
\begin{itemize}
\item Repita el paso anterior cambiando el clasificador HoeffdingTree por
un clasificador HoeffdingTree adaptativo.
\end{itemize}
\begin{verbatim}
for seed in 1 2 314 261 832
do
  java -cp moa.jar -javaagent:sizeofag.jar moa.DoTask \
    "EvaluatePrequential \
      -l (drift.SingleClassifierDrift -l trees.HoeffdingAdaptiveTree) \
	  -s (generators.RandomRBFGeneratorDrift -s 0.001 \
	       -r $seed -i $seed -s 0.001 -k 3 -a 7 -n 3) \
       -i 2000000" > Resultados/Reboot/Adap/adap-$seed.csv
done
\end{verbatim}
\captionof{figure}{\label{org24c619b}
"Aprendizaje Online Hoeffding Tree - Concept Drift - Ventana - Adaptive"}


% latex table generated in R 3.3.2 by xtable 1.8-2 package
% Fri May  5 15:24:26 2017
\begin{table}[ht]
\centering
\begin{tabular}{rrrrr}
  \hline
 & Acc & Kappa & Nodes & Depth \\ 
  \hline
Last & 93.58 & 82.22 & 0.00 & 367.80 \\ 
  Mean & 93.25 & 81.98 & 11.77 & 134.18 \\ 
   \hline
\end{tabular}
\end{table}

\subsubsection{Comparación}
\label{sec:org49f0f09}


	\underline{Wilcoxon rank sum test with continuity correction}\\

data:  acc.reboot.hoeff and acc.reboot.adap\\
W = 11, p-value = 0.8345\\
alternative hypothesis: true location shift is not equal to 0\\

\subsection{Comparación final}
\label{sec:orgff9e531}

\begin{itemize}
\item Responda a la siguiente pregunta: ¿Qué diferencias se producen entre
los métodos de los apartados 2.3, 2.4 y 2.5? Explique similitudes y
diferencias entre las diferentes metodologías, y discuta los
resultados obtenidos por cada una de ellas en el flujo de datos
propuesto.
\end{itemize}

\begin{center}
\includegraphics[width=.9\linewidth]{grafico2.pdf}
\end{center}


En la gráfica vemos cómo el modelo adaptativo es en general mejor,
aunque no hay diferencias significativas cuando reinicializa debido a
que se convierte en adaptativo al reaprender el modelo. 
\end{document}
