% Created 2017-05-09 mar 15:03
% Intended LaTeX compiler: pdflatex
\documentclass[11pt]{article}
\usepackage[utf8]{inputenc}
\usepackage[T1]{fontenc}
\usepackage{graphicx}
\usepackage{grffile}
\usepackage{longtable}
\usepackage{wrapfig}
\usepackage{rotating}
\usepackage[normalem]{ulem}
\usepackage{amsmath}
\usepackage{textcomp}
\usepackage{amssymb}
\usepackage{capt-of}
\usepackage{hyperref}
\author{Jacinto Carrasco Castillo - jacintocc@correo.ugr.es}
\date{20 de mayo de 2017}
\title{Trabajo Autónomo I\\\medskip
\large Series Temporales y Minería de Flujo de Datos - Máster DATCOM}
\hypersetup{
 pdfauthor={Jacinto Carrasco Castillo - jacintocc@correo.ugr.es},
 pdftitle={Trabajo Autónomo I},
 pdfkeywords={},
 pdfsubject={},
 pdfcreator={Emacs 25.1.1 (Org mode 9.0.5)}, 
 pdflang={English}}
\begin{document}

\maketitle


\section{Parte teórica:}
\label{sec:org2513b54}
\subsection{Describir qué es una serie temporal.}
\label{sec:org2efa813}

Llamamos serie temporal a un registro de una magnitud observada a lo
largo del tiempo. Normalmente trabajaremos con series formadas por
observaciones que han tenido lugar en intervalos regulares de
tiempo. Estos datos son útiles cuando la magnitud medida varía a lo
largo del tiempo, y más en concreto cuando hay posibilidad de realizar
una predicción.

\subsection{Describir la metodología Box-Jenkins para predicción de series temporales.}
\label{sec:orge796cd7}

La metodología Box-Jenkins se basa en la consideración de un modelo
aditivo para el modelado de una serie temporal. Para ello, se
determina en un primer paso si la serie es estacionaria, esto es,
suponiendo que la posición viene dada por una distribución de
probabilidad, que sus condiciones son constantes a lo largo del
tiempo. En caso de que esto no se dé, debemos estudiar si hay una
tendencia a largo plazo, si hay un patrón que se repita a lo largo del
tiempo (estacionalidad), o ambas. Una vez que hemos sustraído del
modelo la tendencia y la estacionalidad, volveríamos a comprobar si la
serie es estacionaria, para posteriormente proceder a la
identificación del modelo para identificar el orden de la estacionalidad.

\subsection{Citar varias técnicas de modelado de tendencia. Describir con más detalle aquélla utilizada para resolver la práctica.}
\label{sec:org3aca7f3}
\subsection{Citar varias técnicas de modelado de estacionalidad.  Describir con más detalle aquélla utilizada para resolver la práctica.}
\label{sec:org69d0c09}
\subsection{Describir el proceso para obtener los parámetros de un modelo ARIMA.}
\label{sec:org4a90804}


\section{Parte práctica.}
\label{sec:orgb300e90}
\subsection{Esquematizar los pasos seguidos para conseguir la predicción final (un pequeño resumen de los pasos + dibujo/esquema del proceso).}
\label{sec:org2e86446}



\subsection{Describir y justificar si la serie ha necesitado preprocesamiento. Incluir código en \texttt{R} para realizar esta acción (en su caso).}
\label{sec:org20fe0c6}


\begin{verbatim}
# Cargamos la biblioteca tseries
library(tseries)
library(ggplot2)

# Comenzamos por la lectura de los datos 
serie <- scan("SerieTrabajoPractico.dat")
df <- data.frame(index = seq(1, along.with = serie), serie)
ggplot(df,
       aes(x = index, y = serie)) +
    geom_line()
\end{verbatim}
\captionof{figure}{Figura inicial}

\begin{center}
\includegraphics[width=.9\linewidth]{fig_inicial.png}
\end{center}

Por la gráfica que obtenemos, podemos observar que podría haber una
estacionalidad cada 6 observaciones de la temporalidad, por tanto
consideramos la serie de manera inicial como objeto de la clase \texttt{ts}
con estacionalidad 6.

\begin{verbatim}
# Posible estacionalidad de 6
serie.ts <- ts(serie, frequency = 6)
# Visualizamos la descomposición
plot(decompose(serie.ts))
\end{verbatim}
\captionof{figure}{\label{org3034064}
Descomposición estacionalidad 6}

\begin{center}
\includegraphics[width=.9\linewidth]{fig_decompose.png}
\end{center}

\begin{center}
\includegraphics[width=.9\linewidth]{fig_decompose.png}
\end{center}

En cuanto a la serie obtenida podemos comentar: 
\begin{itemize}
\item La serie tiene una varianza constante, con lo que no será necesario
realizar ningún tipo de preprocesamiento en este sentido.
\end{itemize}


\subsection{Describir y justificar si la serie ha necesitado eliminación de tendencia. Incluir código en \texttt{R} para realizar esta acción (en su caso).}
\label{sec:orgd7d1c35}
\begin{itemize}
\item La serie no presenta una tendencia muy significativa y la media en
distintos subintervalos del período considerado permanece estable,
con lo que no diremos que existe una tendencia a tener en cuenta.
\end{itemize}


\subsection{Describir y justificar si la serie ha necesitado eliminación de estacionalidad. Incluir código en \texttt{R} para realizar esta acción (en su caso).}
\label{sec:org5649930}

\begin{itemize}
\item En la Figura \ref{org3034064} observamos que hay una componente
estacional es muy marcada, con lo que será necesario eliminar esta
estacionalidad. Puesto que hemos considerado una periodicidad de 6
observaciones, consideraremos 6 meses de test, esto es, el segundo
semestre de 2015.
\end{itemize}

\begin{verbatim}
# Dividimos la serie en train y test.
n.test <- 6
index.tra <- seq(1, length(serie.ts) - n.test)
serie.tra <- serie.ts[index.tra]
serie.tst <- serie.ts[-index.tra]
df$tra <- c(rep("tra", length(serie.tra)), rep("tst", n.test))
ggplot(df, aes(x=index, y = serie, color = tra)) + geom_line()
# Trabajaremos en adelante con los datos de entrenamiento.
\end{verbatim}
\captionof{figure}{División en datos de entrenamiento y test}

\begin{center}
\includegraphics[width=.9\linewidth]{train-test.png}
\end{center}

\begin{verbatim}
# Asumimos estacionalidad 6
matrix.tra <- matrix(serie.tra, ncol = 6, byrow=T)
estacionalidad <- apply(matrix.tra, 2, mean)
serie.tra.SinEst <- serie.tra - estacionalidad
serie.tst.SinEst <- serie.tst - estacionalidad
\end{verbatim}

\begin{verbatim}
library(reshape2)
df.est <- data.frame(index.tra, serie.tra, estacionalidad)
df.est <- melt(df.est, id.vars = "index.tra")
ggplot(df.est, aes(x = index.tra, y = value, color = variable)) + geom_line()
\end{verbatim}

\begin{center}
\includegraphics[width=.9\linewidth]{est.png}
\end{center}


\subsection{Describir y justificar si la serie ha necesitado algún proceso para hacerla estacionaria. Incluir código en \texttt{R} para realizar esta acción (en su caso).}
\label{sec:org593f867}


\begin{verbatim}
acf(serie.tra.SinEst)
\end{verbatim}

\begin{center}
\includegraphics[width=.9\linewidth]{ACF_est.png}
\end{center}


\begin{verbatim}
pacf(serie.tra.SinEst)
\end{verbatim}

\begin{center}
\includegraphics[width=.9\linewidth]{PACF_est.png}
\end{center}



\begin{verbatim}
adf.test(serie.tra.SinEst)
\end{verbatim}

\begin{verbatim}

	Augmented Dickey-Fuller Test

data:  serie.tra.SinEst
Dickey-Fuller = -3.5495, Lag order = 4, p-value = 0.04451
alternative hypothesis: stationary
\end{verbatim}

El test de ADF nos arroja un p-valor menor que 0.05 así que podemos
rechazar la hipótesis nula de la no estacionariedad de la
serie. Además, vemos que tanto la gráfica de la autocorrelación
como la de la autocorrelación parcial convergen a 0 rápidamente y
no hay una clara autocorrelación con valores más alejados.  Por lo
tanto, asumimos que la serie ya es estacionaria y por lo tanto no ha
sido necesaria ninguna diferenciación.

\subsection{Describir y justificar cómo se han obtenido los parámetros del modelo ARIMA. Incluir código en R para realizar esta acción.}
\label{sec:org60479c9}


El modelo ARIMA obtenido es \textbf{ARIMA(1,0,0)}, puesto que a partir del
valor 1 podemos considerar que los coeficientes de autocorrelación son
0, a excepción del valor 13 que supera ligeramente el
umbral como vemos en la Figura \ref{org8406a3b}. Igualmente, en la Figura
\ref{orge5792a5} observamos que salvo también el valor 13, el 1 es el único
que supera el umbral. 


\subsection{En el caso de existir más de un modelo inicial planteado, justificar cómo se ha llegado a la toma de decisiones para selección del mejor modelo. Incluir código en \texttt{R} para realizar esta acción (en su caso).}
\label{sec:org926a284}
\subsection{Describir cómo se han obtenido los valores predichos para la serie. Incluir código en \texttt{R} para realizar esta acción.}
\label{sec:orga0feaec}
\begin{verbatim}
modelo <- arima(serie.tra.SinEst, order = c(1,0,0))
valores.ajustados <- estacionalidad + modelo$residuals
\end{verbatim}


\begin{verbatim}
predicciones <- predict(modelo, n.ahead = n.test)$pred
\end{verbatim}

\begin{verbatim}
error.tra <- sum(modelo$residuals^2)
print(error.tra)
error.tst <- sum((predicciones-serie.tst.SinEst)^2)
print(error.tst)
\end{verbatim}

\begin{verbatim}
[1] 1.058235
[1] 0.2921816
\end{verbatim}

\begin{verbatim}
df.results <- data.frame(index=seq(1,length(serie.ts)),
			 serie.ts,
			 pred = c(valores.ajustados, predicciones + estacionalidad),
			 type = c(rep("tra", length(serie.tra)),rep("tst",n.test)))
df.results <- melt(df.results, id.vars = c("index","type"))
ggplot(df.results, aes(y=value,x=index,color=variable,linetype=type))+geom_line()
\end{verbatim}

\begin{center}
\includegraphics[width=.9\linewidth]{results.png}
\end{center}
\end{document}