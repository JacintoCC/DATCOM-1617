% Created 2017-04-08 sáb 13:14
% Intended LaTeX compiler: pdflatex
\documentclass[11pt]{article}
\usepackage[utf8]{inputenc}
\usepackage[T1]{fontenc}
\usepackage{graphicx}
\usepackage{grffile}
\usepackage{longtable}
\usepackage{wrapfig}
\usepackage{rotating}
\usepackage[normalem]{ulem}
\usepackage{amsmath}
\usepackage{textcomp}
\usepackage{amssymb}
\usepackage{capt-of}
\usepackage{hyperref}
\bibliographystyle{plain}
\usepackage[margin=2cm]{geometry}
\author{Jacinto Carrasco Castillo}
\date{}
\title{Algoritmos de Inteligencia Computacional para abordar problemas de detección de outliers.\\\medskip
\large Propuesta de proyecto de investigación}
\hypersetup{
 pdfauthor={Jacinto Carrasco Castillo},
 pdftitle={Algoritmos de Inteligencia Computacional para abordar problemas de detección de outliers.},
 pdfkeywords={},
 pdfsubject={},
 pdfcreator={Emacs 25.1.1 (Org mode 9.0.5)}, 
 pdflang={English}}
\begin{document}

\maketitle

\section{Descripción del problema}
\label{sec:orgac60c27}


Las nuevas tecnologías y su implantación en
el sector de la comunicación o los negocios, un flujo creciente de
datos precisa ser analizado para la extracción de esta información.

En algunas situaciones la generación de un modelo común de
comportamiento no es suficiente para la resolución de un problema sino
que necesitamos detectar y obtener conocimiento de los datos que son
considerados anómalos, es decir, observaciones que se desvían tanto de
las demás muestras que hacen pensar que han sido generadas por
otro mecanismo diferente (\cite{Hawkins80identification}).  Debido a la
naturaleza del problema, no se pueden aplicar las habituales técnicas
de minería de datos (\cite{Souiden16survey}).

Posibles aplicaciones reales de este problema son la monitorización de
sistemas, la detección de fraude en transacciones bancarias, la
seguridad en las telecomunicaciones o sistemas de vigilancia
(\cite{Forestiero16self}).  Como se puede ver, hay una importante
cantidad de distintos problemas que se pueden abordar desde estas
técnicas y requieren de distintas modelizaciones según la naturaleza y
tipología de los datos disponibles, como por ejemplo datos temporales
(\cite{Gupta14outlier}) o trayectorias anómalas (\cite{Lee2008trajectory}).

De manera general podemos mencionar los distintos enfoques que se han
aplicado a la detección de anomalías: 
\begin{itemize}
\item Basados en distancias: Usan la distancia a sus vecinos detectar las
anomalías. Existen varios modelos que explotan este método
(\cite{angiulli05outlier}).
\item Basados en densidad: Las técnicas anteriores tienen el problema de
la posibilidad de que se detecten como anomalías valores en regiones
poco densas y cuyos puntos están a distancias mayores a las
habituales y que sin embargo no constituyan valores extraños
(\cite{Breunig00lof}),
\item Basados en clustering: Entre las técnicas no supervisadas,
podríamos considerar la construcción previa de clusters para agrupar
datos similares, y que los considerados anómalos sean aquellos que
no casen dentro de ninguno de los agrupamientos.
\item Basados en test estadísticos: Asumiendo ciertas propiedades de la
distribución de los datos, estimamos la probabilidad de
pertenencia de cada dato a la población para clasificar como
anomalías los valores que arrojen una probabilidad muy baja de
pertenencia a la distribución de los datos.
\item Basados en clasificación: En este método de detección supervisada de
anomalías, enfocamos el problema como uno de clasificación
muydesbalanceada y que necesitaría aplicar técnicas específicas para
esta situación.
\item Basado en ángulos. Para reducir el problema de la alta
dimensionalidad de los datos existe la detección de anomalías
mediante los ángulos que forman las diferentes observaciones tomadas
como vectores (\cite{Kriegel08angle}).
\end{itemize}

\section{Descripción del proyecto de investigación.}
\label{sec:orgca0f203}

Como hemos visto, existen numerosos enfoques aplicados al problema
  para problemas o tipos de datos específicos. Así pues, el objetivo
  principal de este trabajo será la modelización de la generalización
  de técnicas de detección de anomalías para distintos escenarios,
  incluyendo la detección de anomalías en conjuntos de datos de gran
  tamaño para lo que será necesaria la implementación distribuida de
  estas técnicas. Debido a la reciente expansión a diferentes ámbitos de la
  investigación este campo, existen pocos conjuntos de datos y métodos
  de comparación de las diferentes técnicas (\cite{Campos16evaluation}),
  por lo que se tendrá en cuenta la aplicación de técnicas
  estadísticas para la evaluación de las técnicas de detección de
  anomalías. Posteriormente se podrá realizar la aplicación de estas
  técnicas para el preprocesamiento en problemas tradicionales
  mediante la reducción de ruido (\cite{Liu04line}). En esta dirección
  existe también propuestas que usan la detección de anomalías para la
  reducción de incertidumbre en los datos.  

\section{Planificación y metodología a seguir}
\label{sec:org3a4e830}

\begin{itemize}
\item En una primera fase pretendemos realizar una revisión de los
diferentes enfoques realizados para la detección de datos anómalos y
su eficacia conforme aumenta el volumen de datos.
\item La tarea fundamental consistirá en el diseño e implementación de
algoritmos de detección de anomalías para Big Data que sean
aplicables a distintos tipos de datos como series temporales o
\item Para comprobar la eficacia de los algoritmos se aplicará estos
modelos en distintos escenarios y conjuntos de datos con anomalías y
obtención de los resultados. Se usará también para la
reducción de ruido en problemas de clasificación.
\item Se realizará una comparación de los resultados con otras técnicas
relevantes presentes en la literatura.
\item Según los resultados obtenidos se variarán los modelos diseñados
para ajustar su calidad. Este proceso se repetirá hasta la obtención
de unos resultados satisfactorios.
\end{itemize}

\section{Conclusiones que se esperan obtener}
\label{sec:org9f2a410}

Con la realización de este proyecto se espera obtener una serie de
algoritmos aplicables en distintos contextos que sean útiles en la
detección de anomalías para conjuntos de datos que requieran de
técnicas de Big Data.

\bibliography{bibliography}
\end{document}
