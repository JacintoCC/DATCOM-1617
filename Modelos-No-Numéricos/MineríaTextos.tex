% Created 2017-06-09 vie 22:45
% Intended LaTeX compiler: pdflatex
\documentclass[11pt]{article}
\usepackage[utf8]{inputenc}
\usepackage[T1]{fontenc}
\usepackage{graphicx}
\usepackage{grffile}
\usepackage{longtable}
\usepackage{wrapfig}
\usepackage{rotating}
\usepackage[normalem]{ulem}
\usepackage{amsmath}
\usepackage{textcomp}
\usepackage{amssymb}
\usepackage{capt-of}
\usepackage{hyperref}
\usepackage{minted}
\author{Jacinto Carrasco Castillo}
\date{9 de junio}
\title{Minería de textos en KNIME\\\medskip
\large Modelos de datos no numéricos}
\hypersetup{
 pdfauthor={Jacinto Carrasco Castillo},
 pdftitle={Minería de textos en KNIME},
 pdfkeywords={},
 pdfsubject={},
 pdfcreator={Emacs 25.2.1 (Org mode 9.0.7)}, 
 pdflang={English}}
\begin{document}

\maketitle

\section{Descripción de los documentos}
\label{sec:org84e2938}

Para la práctica de minería de textos en KNIME se ha usado una
colección de documentos PDF relativos al uso de test estadísticos para
la comparación de resultados de algoritmos de aprendizaje automático
en problemas de clasificación. Estos artículos podríamos clasificarlos
en varios tipos
\begin{itemize}
\item Estadística frecuentista
\begin{itemize}
\item Metodologías paramétricas
\item Metodologías no paramétricas
\begin{itemize}
\item Test permutacionales com subtipo de los métodos no paramétricos.
\end{itemize}
\end{itemize}
\item Estadística bayesiana.
\end{itemize}


\section{Workflow}
\label{sec:org74724de}

\begin{figure}[htbp]
\centering
\includegraphics[width=.9\linewidth]{./Workflow.png}
\caption{\label{fig:org8427546}
Workflow}
\end{figure}

Los métodos para preprocesar los documentos han sido:
\begin{itemize}
\item POS tagger.
\item Eliminación de la puntuación y filtro por números.
\item Filtro de \emph{stop words}.
\item Conversión a minúsculas.
\item \emph{Porter stemmer}.
\item \emph{Bag of words}.
\item Filtro por filas: Una vez que tenemos la bolsa de palabras y una
columna con los términos, vemos que aún siguen quedando términos con
signos como \(\delta\), \(\sum\)\ldots{} así que filtramos con una expresión
regular para quedarnos únicamente con palabras.
\end{itemize}

\section{Resultados}
\label{sec:orgd77498d}

Mostramos en la \hyperref[fig:orgda7cefc]{Imagen 2} la nube de palabras (aunque hemos mostrado
las racíces de las palabras para evitar duplicidades en los
términos). Observamos como \emph{algorithm, test, bootstrap, dataset, rank,
CV} son términos relevantes.


\begin{figure}[htbp]
\centering
\includegraphics[width=.9\linewidth]{./Tagcloud.png}
\caption{\label{fig:orgda7cefc}
Tagcloud}
\end{figure}


Una vez que hemos realizado la nube de términos, se genera un vector
por documento para los términos más relevantes. Aplicamos el algoritmo
de \(k\) medias, y mostramos en la \hyperref[fig:orge0f1e51]{Figura 3} para algunos de los términos más relevantes
la matriz relativa a estos términos y la posición de los
artículos. Observamos que la mayoría de los artículos han sido
asignados a una clase, con lo que no se ha obtenido la partición
esperada según el tipo de técnicas estadísticas usadas para la
comparación de algoritmos. Sin embargo sí que vemos que separa los
documentos que tratan sobre test basados en permutaciones.

\begin{figure}[htbp]
\centering
\includegraphics[width=.9\linewidth]{./Scatter.png}
\caption{\label{fig:orge0f1e51}
Scatter matrix para los términos "test","statistic" y "permut"}
\end{figure}
\end{document}