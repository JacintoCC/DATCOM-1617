% Created 2017-06-11 dom 21:05
% Intended LaTeX compiler: pdflatex
\documentclass[11pt]{article}
\usepackage[utf8]{inputenc}
\usepackage[T1]{fontenc}
\usepackage{graphicx}
\usepackage{grffile}
\usepackage{longtable}
\usepackage{wrapfig}
\usepackage{rotating}
\usepackage[normalem]{ulem}
\usepackage{amsmath}
\usepackage{textcomp}
\usepackage{amssymb}
\usepackage{capt-of}
\usepackage{hyperref}
\usepackage{minted}
\author{Jacinto Carrasco Castillo. DNI 32056356Z \\  e-mail: jacintocc@correo.ugr.es}
\date{}
\title{Análisis y visualización básica de una red social con \emph{Gephi}\\\medskip
\large Modelos de ciencia de datos no numéricos \\  Máster DATCOM. Curso 16-17.}
\hypersetup{
 pdfauthor={Jacinto Carrasco Castillo. DNI 32056356Z \\  e-mail: jacintocc@correo.ugr.es},
 pdftitle={Análisis y visualización básica de una red social con \emph{Gephi}},
 pdfkeywords={},
 pdfsubject={},
 pdfcreator={Emacs 25.2.1 (Org mode 9.0.7)}, 
 pdflang={Sp}}
\begin{document}

\maketitle
\tableofcontents

\newpage

\section{Red}
\label{sec:orgcc443b3}

\subsection{Descripción de la red}
\label{sec:org89545bb}

La red utilizada ha sido la formada por los personajes de Los Simpson
en la primera temporada, donde dos personajes están unidos si aparecen
en el mismo capítulo. Esta red se puede descargar en formato \texttt{.gephi}
desde el siguiente \href{https://gist.github.com/cpudney/6dfc60b2cf1d4d390e2e}{enlace}. Mostramos la red en la Figura \ref{fig:orge1d5d99}.
Esta gráfica usa la centralidad para fijar el tamaño de los nodos y el
grado para seleccionar el color.


\begin{figure}[htbp]
\centering
\includegraphics[width=.9\linewidth]{./GephiOutput/Red.png}
\caption{\label{fig:orge1d5d99}
Red completa}
\end{figure}
En esta red los nodos más destacados son los de la familia Simpson,
como veremos más adelante. 



\subsection{Tabla con valores de las medidas}
\label{sec:org60422d0}

\begin{table}[htbp]
\caption{\label{tab:orgbfc146c}
Medidas del grafo}
\centering
\begin{tabular}{lr}
Medida & Valor\\
\hline
Número de nodos \(N\) & 238\\
Número de enlaces \(L\) & 6192\\
Número máximo de enlaces \(L_{max}\) & 28203\\
Densidad del grafo \(L/L_{max}\) & 0.22\\
\hline
Grado medio \(\langle k \rangle\) & 52.034 / 64.252\\
Diámetro \(d_{max}\) & 2\\
Distancia media & 1.78\\
Coeficiente medio de clustering \(\langle C \rangle\) & 0.881\\
\hline
Número de componentes conexas & 1\\
Número de nodos componente gigante (y \%) & 238,  100\%\\
Número de aristas componente gigante (y \%) & 6192, 100\%\\
\end{tabular}
\end{table}

\subsection{Gráficos con las distribuciones}
\label{sec:org1bb15db}

\begin{enumerate}
\item Grado medio
\label{sec:org9f613c0}
Mostramos en la Figura \ref{fig:org520fdb0} la distribución de número de nodos
por grado.

\begin{figure}[htbp]
\centering
\includegraphics[width=.9\linewidth]{./GephiOutput/AvgDegree/degree-distribution.png}
\caption{\label{fig:org520fdb0}
Distribución número de nodos por grado}
\end{figure} 

\begin{enumerate}
\item Grado medio ponderado
\label{sec:org09ea5d4}

En la Figura \ref{fig:org618bbbe} se muestra la distribución de número de
nodos por la ponderación de grados. 

\begin{figure}[htbp]
\centering
\includegraphics[width=.9\linewidth]{./GephiOutput/AvgWeightDegree/w-degree-distribution.png}
\caption{\label{fig:org618bbbe}
Distribución número de nodos por grado con pesos}
\end{figure}
\end{enumerate}


\item Distribución de distancias
\label{sec:org0e2cc79}

Incluimos en la Figura \ref{fig:orgabd3dd8} la distribución de la
centralidad de los nodos. Observamos tanto en esta gráfica como en las
anteriores la existencia de los cinco nodos conectados con todos los
demás.

\begin{figure}[htbp]
\centering
\includegraphics[width=.9\linewidth]{./GephiOutput/Diameter/Closeness Centrality Distribution.png}
\caption{\label{fig:orgabd3dd8}
Distribución de medida de centralidad}
\end{figure}

\item Coeficiente medio de clustering
\label{sec:org26b26df}

En la Figura \ref{fig:org41cc9ee} observamos la distribución de número de
nodos por coeficiente de clustering.

\begin{figure}[htbp]
\centering
\includegraphics[width=.9\linewidth]{./GephiOutput/AvgClusteringCoefficient/clustering-coefficient.png}
\caption{\label{fig:org41cc9ee}
Distribución de número de nodos por coeficiente de clustering}
\end{figure}
\end{enumerate}


\section{Análisis}
\label{sec:org53aebc9}

Por la característica de la red y al aparecer todos los componentes de
la familia Simpson en todos los capítulos, éstos están conectados con
todos los demás nodos de la ciudad. Entonces, tenemos que de cualquier
personaje podríamos llegar a cualquier otro sin más que pasar porp
cualquier personaje central. 

Como vemos en la Figura \ref{fig:org520fdb0}, el menor grado es de 20 y lo
poseen algunos nodos. Esto significa que algunos nodos aparecen
únicamente en un capítulo con pocos personajes. Aunque la moda es
también cercana a 20, por lo que hay algún capítulo en el que hay más
personajes que sólo aparecen en ese capítulo. Conforme aumentamos el
grado, vemos que va habiendo menos nodos, hasta llegar al grado
máximo, donde nos encontramos con los cinco nodos centrales. 

En cuanto a la medida de clustering, vemos que hay un muy alto número
de puntos con coeficiente de clustering 1. Estos nodos serán todos
aquellos que salgan en un único capítulo, pues todos sus vecinos
también estarán interconectados y formarán un clique entre ellos. 


\section{Actores centrales}
\label{sec:org81f1cd9}

\subsection{Tabla con actores centrales y medidas}
\label{sec:orgca66121}

Para mostrar a los actores principales hemos agrupado a los cinco
personajes (Marge, Homer, Bart, Lisa y Maggie) de la familia
Simpson, pues tienen los mismos valores y así mostramos un poco más de
información sobre la red. 

\begin{center}
\begin{tabular}{lrrrr}
Personaje & Grado & Intermediación & Cercanía & Vector propio\\
\hline
Familia Simpson & 237 & 2479.63 & 1 & 1\\
Moe Syzlak & 166 & 792.58 & 0.7695 & 0.8286\\
Barney Gumble & 166 & 792.58 & 0.7695 & 0.8286\\
Lewis Clark & 163 & 674.72 & 0.7621 & 0.8216\\
Milhouse & 152 & 551.28 & 0.7362 & 0.7896\\
\end{tabular}
\end{center}

\section{Detección de comunidades}
\label{sec:orgede750a}

El estudiante escogerá distintos valores para \emph{resolución} que
determina el número de comunidades. Deberá perseguir la obtención de
un número razonableque permita realizar un buen análisis de la
estructura de comunidades obtenida. Mostrará los valores de la medida
de modularidad asociados a cada particionamiento realizado y analizara
la composición de las comunidades generadas para determinar si tienen
algún tipo de influencia en la estructura de la red.



\subsection{Resolución 1}
\label{sec:org47aa166}

Tanto en la Figura \ref{fig:orge1d5d99} como en la Figura \ref{fig:orgd7d5a91} se observan
distintas zonas (y colores en ésta segunda figura) que guardan
relación con los capítulos de la temporada, donde personajes que
aparecen en el mismo capítulo (y sólo en ese capítulo) están situados
más cerca y con el mismo color.

\begin{figure}[htbp]
\centering
\includegraphics[width=.9\linewidth]{./GephiOutput/RedComunidades.png}
\caption{\label{fig:orgd7d5a91}
Distribución de tamaño de comunidades. Resolución = 1.}
\end{figure}


\subsection{Resolución 1.5}
\label{sec:org3b735b3}

Si aumentamos el valor de resolución, se reduce el número de grupos y
surge un grupo central con los personajes de mayor relevancia que
acoge a varios de los grupos que relacionamos con capítulos
independientes, como observamos en la Figura \ref{fig:org8e51ae5}.


\begin{figure}[htbp]
\centering
\includegraphics[width=.9\linewidth]{./GephiOutput/RedComunidades1-5.png}
\caption{\label{fig:org8e51ae5}
Distribución de tamaño de comunidades. Resolución = 1.5 .}
\end{figure}


\subsection{Resolución 0.5}
\label{sec:orgf2fb671}

Si reducimos el valor de \emph{resolución} esperamos encontrarnos con
distintos clusters según el capítulo. Nos encontramos con resultado
poco esperados, como que la familia Simpson se ha dividido y Lisa
y Maggie pertenecen a distintos clusters que el resto de la familia
(pese a tener los mismos enlaces), y nos encontramos un cluster
formado únicamente por dos nodos (en color marrón oscuro): Los de Patti y Selma Bouvier, las
hermanas de Marge, con lo que el algoritmo ha sabido identificar un
conjunto que es inseparable. 

\begin{figure}[htbp]
\centering
\includegraphics[width=.9\linewidth]{./GephiOutput/RedComunidades05.png}
\caption{\label{fig:orgadcc2b2}
Distribución de tamaño de comunidades. Resolución = 0.5 .}
\end{figure}
\end{document}